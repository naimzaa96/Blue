# Problem Setup
## Understanding the difference between analytical and numerical modeling using eulers method.

First we solve for $\Delta N_{A}$. Given the expression 



$$\frac{dN_{A}}{dt} = - \frac{N_{A}(t)}{\tau_{A}}$$

$$\Rightarrow \Delta N_{A} = - \frac{N_{A}(t)}{\tau_{A}}\Delta t$$

We also know that 

$$\Delta N_{A} = N_{A}(t+\Delta t) - N_{A}(t)$$

$$\Rightarrow  N_{A}(t+\Delta t)  = N_{A}(t) +\Delta N_{A} $$

Pluggin in $\Delta N_{A}$ gives:

$$\boxed{N_{A}(t+\Delta t) = N_{A}(t) - \frac{N_{A}(t)}{\tau_{A}}\Delta t}$$

This is enough to solve the expression.

First lets set some constants:

We are using Euler's method so at every given $N_{A}(t)$ we will have a corresponding **slope**. The shorter the time intervals are when propogating forward with these slopes the more accurate the estimation. That being said, we have to choose how long these intervals are and how many times we want to step forward using this chosen value. We can do something a bit simpler by just stating some overall time and dividing that by the time intervals to get a number of iterations. To our benefit, $\tau_{A}$ is known and we simply pick a value to call it. 

**Working in units of $\tau_{A}$**.



To we solve for $\Delta N_{B}$. Given the expression 



$$\frac{dN_{B}}{dt} = \frac{N_{A}}{\tau_{A}}- \frac{N_{B}(t)}{\tau_{B}}$$ 

$$\Rightarrow \Delta N_{B} = \frac{N_{A}(t)}{\tau_{A}}\Delta t - \frac{N_{B}(t)}{\tau_{B}}\Delta t$$

We also know that 

$$\Delta N_{B} = N_{B}(t+\Delta t) - N_{B}(t)$$

$$\Rightarrow  N_{B}(t+\Delta t)  = N_{B}(t) +\Delta N_{B} $$

Pluggin in $\Delta N_{A}$ gives:

$$\boxed{N_{B}(t+\Delta t) = N_{B}(t)\left(1-\frac{\Delta t}{\tau_{B}}\right) + \frac{N_{A}(t)}{\tau_{A}}\Delta t}$$

This is enough to solve the expression.



---
